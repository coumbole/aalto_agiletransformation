% ---------------------------------------------------------------------
% -------------- PREAMBLE ---------------------------------------------
% ---------------------------------------------------------------------

\documentclass[12pt,a4paper,english,oneside]{article}
%\documentclass[lnbip]{svmultln}

\usepackage[utf8]{inputenc}
\usepackage[english]{babel}

\usepackage{calc}
\usepackage{natbib}
\usepackage{url}
\usepackage{listings}
\usepackage{hyphenat}

\usepackage{supertabular,array}
\usepackage{lipsum}
\usepackage{booktabs}
\usepackage{graphicx}


% Strike through
\usepackage[normalem]{ulem}

% Punctuation for references
\bibpunct{[}{]}{;}{n}{,}{,}

% correct bad hyphenation here
\hyphenation{op-tical net-works semi-conduc-tor}


\usepackage{multibib}
\newcites{gen}{References}
\usepackage{bibentry}


% ---------------------------------------------------------------------
% -------------- DOCUMENT ---------------------------------------------
% ---------------------------------------------------------------------

\begin{document}

\selectlanguage{english}

\mainmatter


\title{Adopting agile in the large: motivations, success factors and challenges}

\titlerunning{Adopting agile in the large}

\author{Kim Dikert \and Maria Paasivaara \and Casper Lassenius}
\authorrunning{Kim Dikert et al.}   % abbreviated author list (for running head)

% list of authors for the TOC (use if author list has to be modified)
\tocauthor{Kim Dikert, Maria Paasivaara, Casper Lassenius}

\institute{Aalto University, School of Science,
Department of Computer Science and Engineering
\\ \email{kim-karol.dikert@aalto.fi}, \email{maria.paasivaara@aalto.fi}, \email{casper.lassenius@aalto.fi} }

\maketitle


% ---------------------------------------------------------------------
% -------------- ABSTRACT AND TEXT ------------------------------------
% ---------------------------------------------------------------------

\begin{abstract}

Agile methods have become an appealing alternative for companies striving to
improve their performance, but the methods are originally designed for small and
individual teams. This poses challenges when introducing agile to large
organizations where development teams can not be autonomous. We reviewed 30
experience reports and case studies describing the process of taking agile
methods into use in large organizations. The primary motive for agile adoption
is the need to improve performance. For a successful transformation the entire
organization must be aligned and all organizational units must participate.
Creating an agile community in the organization will help the change to stick.
Typical problems are misunderstanding of the agile model, and difficulties
coordinating multiple agile teams.

\keywords {agile, transformation, large scale, adopting agile}
\end{abstract}


% ---------------------------------------------------------------------
\section{Introduction}


% ---------------------------------------------------------------------
\section{Background}
\label{sec:background}


% ---------------------------------------------------------------------
\section{Research method}
\label{sec:method}


% ---------------------------------------------------------------------
\section{Results}
\label{sec:results}


% ---------------------------------------------------------------------
\section{Limitations and future research}
\label{sec:conclusion}


% ---------------------------------------------------------------------
% -------------- BIBLIOGRAPHY -----------------------------------------
% ---------------------------------------------------------------------

%\small
\fontsize{9pt}{10pt}\selectfont

\bibliographystyle{plain}
\bibliographystylegen{plain}

\bibliographygen{sources}

\nobibliography{sources_s}

\subsection*{Primary studies}


\begin{supertabular}{ l p{11.4cm} }
    {[}S1{]} & \bibentry{S1} \\  \shrinkheight{-1cm}
    {[}S2{]} & \bibentry{S2} \\ 
    {[}S3{]} & \bibentry{S3} \\ 
    {[}S4{]} & \bibentry{S4} \\ 
    {[}S5{]} & \bibentry{S5} \\ 
    {[}S6{]} & \bibentry{S6} \\ 
    {[}S7{]} & \bibentry{S7} \\ 
    {[}S8{]} & \bibentry{S8} \\ 
    {[}S9{]} & \bibentry{S9} \\
    {[}S10{]} & \bibentry{S10} \\ 
    {[}S11{]} & \bibentry{S11} \\ 
    {[}S12{]} & \bibentry{S12} \\  \shrinkheight{-3cm}
    {[}S13{]} & \bibentry{S13} \\ 
    {[}S14{]} & \bibentry{S14} \\ 
    {[}S15{]} & \bibentry{S15} \\ 
    {[}S16{]} & \bibentry{S16} \\ 
    {[}S17{]} & \bibentry{S17} \\ 
    {[}S18{]} & \bibentry{S18} \\ 
    {[}S19{]} & \bibentry{S19} \\ 
    {[}S20{]} & \bibentry{S20} \\ 
    {[}S21{]} & \bibentry{S21} \\ 
    {[}S22{]} & \bibentry{S22} \\ 
    {[}S23{]} & \bibentry{S23} \\ 
    {[}S24{]} & \bibentry{S24} \\ 
    {[}S25{]} & \bibentry{S25} \\ 
    {[}S26{]} & \bibentry{S26} \\ 
    {[}S27{]} & \bibentry{S27} \\ 
    {[}S28{]} & \bibentry{S28} \\ 
    {[}S29{]} & \bibentry{S29} \\ 
    {[}S30{]} & \bibentry{S30} \\ 
\end{supertabular}\end{document}
