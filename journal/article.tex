% ---------------------------------------------------------------------
% -------------- PREAMBLE ---------------------------------------------
% ---------------------------------------------------------------------

%\documentclass[12pt,a4paper,english,oneside]{article}
\documentclass[lnbip]{svmultln}

\usepackage[utf8]{inputenc}
\usepackage[english]{babel}

\usepackage{calc}
\usepackage{natbib}
\usepackage{url}
\usepackage{listings}
\usepackage{hyphenat}

\usepackage{supertabular,array}
\usepackage{lipsum}
\usepackage{booktabs}
\usepackage{graphicx}


% Strike through
\usepackage[normalem]{ulem}

% Punctuation for references
\bibpunct{[}{]}{;}{n}{,}{,}

% correct bad hyphenation here
\hyphenation{op-tical net-works semi-conduc-tor}


\usepackage{multibib}
\newcites{gen}{References}
\usepackage{bibentry}


% ---------------------------------------------------------------------
% -------------- DOCUMENT ---------------------------------------------
% ---------------------------------------------------------------------

\begin{document}

\selectlanguage{english}

\mainmatter


\title{Adopting agile in the large: motivations, success factors and challenges}

\titlerunning{Adopting agile in the large}

\author{Kim Dikert \and Maria Paasivaara \and Casper Lassenius}
\authorrunning{Kim Dikert et al.}   % abbreviated author list (for running head)

% list of authors for the TOC (use if author list has to be modified)
\tocauthor{Kim Dikert, Maria Paasivaara, Casper Lassenius}

\institute{Aalto University, School of Science,
Department of Computer Science and Engineering
\\ \email{kim-karol.dikert@aalto.fi}, \email{maria.paasivaara@aalto.fi}, \email{casper.lassenius@aalto.fi} }

\maketitle


% ---------------------------------------------------------------------
% -------------- ABSTRACT AND TEXT ------------------------------------
% ---------------------------------------------------------------------

\begin{abstract}

Agile methods have become an appealing alternative for companies striving to
improve their performance, but the methods are originally designed for small and
individual teams. This poses challenges when introducing agile to large
organizations where development teams can not be autonomous. We reviewed 30
experience reports and case studies describing the process of taking agile
methods into use in large organizations. The primary motive for agile adoption
is the need to improve performance. For a successful transformation the entire
organization must be aligned and all organizational units must participate.
Creating an agile community in the organization will help the change to stick.
Typical problems are misunderstanding of the agile model, and difficulties
coordinating multiple agile teams.

\keywords {agile, transformation, large scale, adopting agile}
\end{abstract}


% ---------------------------------------------------------------------
\section{Introduction}

--> Introductory paragraph

--> Motivation for research
-- There are no systematic reviews for this
-- A collection of experience reports can be used for creating a theoretical
   base.

--> This literature review is intended for researchers as a theroretical
   and proven foundation for the real problems that organizations face in
   transformations.
   This is also intended to be directly usable by practicioners, to serve as a
   base of general advice with strong motivation (although every organization
   has unique needs).

% ---------------------------------------------------------------------
\section{Background}
\label{sec:background}


% ---------------------------------------------------------------------
\section{Research method}
\label{sec:method}

\subsection{Research questions}


RQ1 Why are transformations initiated?

RQ2 How do transformations proceed usually; do transformations exhibit patterns?

RQ3 What are typical successes and challenges in an large scale agile
    transformation process?


\subsection{Research process}

The research was conducted as an application of Kitchenham's
\cite{Kitchenham2007} guidelines for systematic literature review. The selection
of primary studies was done as a keyword based database search and manual
filtering of included studies. The filtering process was executed independently
by two researchers (Dikert, Paasivaara), and the process was audited by a third
external researcher (Lassenius). We deviated from Kitchenham's guideline in the
parts of study assessment and data extraction. These deviations are explained
below.


The reason for deviating stydy assessment
--> Original research consisted almost solely of experience reports
--> The fact that experience reports have a subjective standpoint makes it
    unreliable to make quantitative conclusions based on the data
--> We also decided to include separate studies describing the same organization.
    This was useful as it deepened the understanding of each case, and several
    large organizations have produced more than one description of their agile
    transformation.

The fact that the study is based on a quantitative basis
--> Experie reports have a subjective bias
--> The data synthesis must not rely on quantitative measures. This means that
    we also considered topics raised by only a marginal number of reports.
--> It may still be interesting to make some quantitative observations such as
    a particular problem being reported in the majority of reports.


Data extraction was implemented partially using data extraction forms, but also
by a iterative coding approach. (The reason was? ...). Questions for which the
answer was easy to trace (for example, organization size, time of
transformation, and RQ1 why are transformations initiated) were answered by
filling in a data extraction form. For the more complex research questions (RQ2
how does transformation usually proceed, RQ3 which are typical success factors
and challenges) the iterative coding approach was used. The iterative coding
approach...
--> Possibility to find many different viewpoinits for the research questions
--> Lots of items, which would have lead to much diversity in data extraction form 

% ---------------------------------------------------------------------
\section{Results}
\label{sec:results}


% ---------------------------------------------------------------------
\section{Discussion}
\label{sec:discussion}


% ---------------------------------------------------------------------
\section{Limitations and future research}
\label{sec:conclusion}


% ---------------------------------------------------------------------
% -------------- BIBLIOGRAPHY -----------------------------------------
% ---------------------------------------------------------------------

%\small
\fontsize{9pt}{10pt}\selectfont

\bibliographystyle{plain}
\bibliographystylegen{plain}

\bibliographygen{sources}

\nobibliography{sources_s}

\subsection*{Primary studies}


\begin{supertabular}{ l p{11.4cm} }
    {[}S1{]} & \bibentry{S1} \\  \shrinkheight{-1cm}
    {[}S2{]} & \bibentry{S2} \\ 
    {[}S3{]} & \bibentry{S3} \\ 
    {[}S4{]} & \bibentry{S4} \\ 
    {[}S5{]} & \bibentry{S5} \\ 
    {[}S6{]} & \bibentry{S6} \\ 
    {[}S7{]} & \bibentry{S7} \\ 
    {[}S8{]} & \bibentry{S8} \\ 
    {[}S9{]} & \bibentry{S9} \\
    {[}S10{]} & \bibentry{S10} \\ 
    {[}S11{]} & \bibentry{S11} \\ 
    {[}S12{]} & \bibentry{S12} \\  \shrinkheight{-3cm}
    {[}S13{]} & \bibentry{S13} \\ 
    {[}S14{]} & \bibentry{S14} \\ 
    {[}S15{]} & \bibentry{S15} \\ 
    {[}S16{]} & \bibentry{S16} \\ 
    {[}S17{]} & \bibentry{S17} \\ 
    {[}S18{]} & \bibentry{S18} \\ 
    {[}S19{]} & \bibentry{S19} \\ 
    {[}S20{]} & \bibentry{S20} \\ 
    {[}S21{]} & \bibentry{S21} \\ 
    {[}S22{]} & \bibentry{S22} \\ 
    {[}S23{]} & \bibentry{S23} \\ 
    {[}S24{]} & \bibentry{S24} \\ 
    {[}S25{]} & \bibentry{S25} \\ 
    {[}S26{]} & \bibentry{S26} \\ 
    {[}S27{]} & \bibentry{S27} \\ 
    {[}S28{]} & \bibentry{S28} \\ 
    {[}S29{]} & \bibentry{S29} \\ 
    {[}S30{]} & \bibentry{S30} \\ 
\end{supertabular}\end{document}
