% ---------------------------------------------------------------------
% -------------- PREAMBLE ---------------------------------------------
% ---------------------------------------------------------------------

%\documentclass[12pt,a4paper,english,oneside]{article}
\documentclass[preprint,authoryear,12pt]{elsarticle}

\usepackage[utf8]{inputenc}
\usepackage[english]{babel}

\usepackage{calc}
\usepackage{natbib}
\usepackage{url}
%\usepackage{listings}
\usepackage{hyphenat}

\usepackage{supertabular,array}
\usepackage{lipsum}
\usepackage{booktabs}
\usepackage{graphicx}


% Strike through
\usepackage[normalem]{ulem}

% Punctuation for references
\bibpunct{[}{]}{;}{n}{,}{,}

% correct bad hyphenation here
\hyphenation{op-tical net-works semi-conduc-tor}


% Multiple bibliographies
%\usepackage{multibib}
%\newcites{gen}{References}
%\usepackage{bibentry}


% ---------------------------------------------------------------------
% -------------- DOCUMENT ---------------------------------------------
% ---------------------------------------------------------------------

\journal{Information and Software Technology}

\begin{document}

% Extra definitions?
\selectlanguage{english}

\begin{frontmatter}

%% use the tnoteref command within \title for footnotes;
%% use the tnotetext command for the associated footnote;
%% use the fnref command within \author or \address for footnotes;
%% use the fntext command for the associated footnote;
%% use the corref command within \author for corresponding author footnotes;
%% use the cortext command for the associated footnote;
%% use the ead command for the email address,
%% and the form \ead[url] for the home page:
%%
%% \title{Title\tnoteref{label1}}
%% \tnotetext[label1]{}
%% \author{Name\corref{cor1}\fnref{label2}}
%% \ead{email address}
%% \ead[url]{home page}
%% \fntext[label2]{}
%% \cortext[cor1]{}
%% \address{Address\fnref{label3}}
%% \fntext[label3]{}

%% use optional labels to link authors explicitly to addresses:
%% \author[label1,label2]{<author name>}
%% \address[label1]{<address>}
%% \address[label2]{<address>}

\title{Adopting agile in the large: motivations, success factors and challenges}

\author{Kim Dikert}
\ead{kim-karol.dikert@aalto.fi}

\author{Maria Paasivaara}
\ead{maria.paasivaara@aalto.fi}

\author{Casper Lassenius}
\ead{casper.lassenius@aalto.fi}

\address{Aalto University, School of Science, Department of Computer Science and Engineering}


% ---------------------------------------------------------------------
% ------------------------- ABSTRACT ----------------------------------
% ---------------------------------------------------------------------

\begin{abstract}

Agile methods have become an appealing alternative for companies striving to
improve their performance, but the methods are originally designed for small and
individual teams. This poses challenges when introducing agile to large
organizations where development teams can not be autonomous. We reviewed 30
experience reports and case studies describing the process of taking agile
methods into use in large organizations. The primary motive for agile adoption
is the need to improve performance. For a successful transformation the entire
organization must be aligned and all organizational units must participate.
Creating an agile community in the organization will help the change to stick.
Typical problems are misunderstanding of the agile model, and difficulties
coordinating multiple agile teams.

\end{abstract}

\begin{keyword}
%% keywords here, in the form: keyword \sep keyword
%% MSC codes here, in the form: \MSC code \sep code
%% or \MSC[2008] code \sep code (2000 is the default)

agile \sep transformation \sep large scale \sep adopting agile

\end{keyword}

\end{frontmatter}


% ---------------------------------------------------------------------
% --------------------------- TEXT ------------------------------------
% ---------------------------------------------------------------------


\section{Introduction}

--> Introductory paragraph

--> Motivation for research
-- There are no systematic reviews for this
-- A collection of experience reports can be used for creating a theoretical
   base.
--> There is interest in large scale [ref XP13 workshop]


--> Why is large scale software development different from smaller scale?

--> This literature review is intended for researchers as a theroretical
   and proven foundation for the real problems that organizations face in
   transformations.
   This is also intended to be directly usable by practicioners, to serve as a
   base of general advice with strong motivation (although every organization
   has unique needs).

% ---------------------------------------------------------------------
\section{Background}
\label{sec:background}


% ---------------------------------------------------------------------
\section{Research method}
\label{sec:method}

\subsection{Research questions}

\begin{itemize}

\item
RQ1: Why are transformations initiated?

\item
RQ2: How do transformations proceed usually; do transformations exhibit patterns?

\item
RQ3: What are typical successes and challenges in an large scale agile
    transformation process?

\end{itemize}

\subsection{Research process}

The research was conducted as an application of Kitchenham's
\cite{Kitchenham2007} guidelines for systematic literature review. The selection
of primary studies was done as a keyword based database search and manual
filtering of included studies. The filtering process was executed independently
by two researchers (Dikert, Paasivaara), and the process was audited by a third
external researcher (Lassenius).

We deviated from Kitchenham's guideline in the parts of study assessment and
data extraction. These deviations are explained below.
--> This is maybe not true! The study assessment can also be done qualitatively.


\subsection{Inclusion criteria}
- pitää olla systemaattinen
- pitää olla toistettava
==> Perustuu enimmäkseen siihen että on 2 tutkijaa toistamassa


There must be more discussion on transformation than a couple of sentences.
--> Esimerkkejä
--> Tasot 1--5 ??

Listing of not included topics. Discussion on these topics are considered not
providing insight to the research questions. Our goal is to address the research
questions, and specifically focus on analyzing the transformation process.
Not included topics:
--> Benefits of agile in general, problems with agile in general
--> Comparing agile and method X
--> Reporting on the current method, without consideration of how the method has
    been evolved (or transformed)
Tämän perusteella pitää siis kertoa itse prosessista: --> Kirjoita auki tämä
kriteeri paremmin.


Jos koko on epäselvä:
--> Tutkijat päättävät keskustelun ja kvalitatiivisen harkinnan perusteella
--> Tähän tarvitaan esimerkit, milloin on otettu mukaan, milloin on karsittu
    pois


% -----------------------


Based on the research questions and intended focus of the research, we defined
an inclusion criteria with three facets: agile software engineering, focus on
transformation, and large organizations. The first facet covers primary studies
focusing on software engineering organizations applying or striving to apply
agile methods. Examples on topics excluded by the first facet are agile
manufacturing, and applying Scrum practices in management boards. The second
facet states that included studies must provide insights relevant to
organizational transformation, and specifically to the research questions.
The third facet underlines the particular contribution we aim to provide with
this research, namely the exclusive focus on large organizations.

A key question in the study was the definition of \emph{large scale}.
--> Ref: the XP conference

--> Case studies ??

A database keyword search was chosen as the search strategy. Based on the
research questions three facets for keywords were chosen: agile software
development, organizational transformation, and large scale. Preliminary
searches showed however that picking keywords with good precision was very
difficult. As a result we excluded the third facet (large scale). Instead of
engineering the search terms, which proved to result in excluding some
relevant articles, we chose an approach of manually filtering the search
results. Still having a substantial part of the matches in the non-relevant  
areas such as agile manufacturing, we decided to limit searches to either be
included in major software publications or include the term ``software''. The
resulting facets and keywords are listed in Table \ref{table:searchterms}. The
databases included in the search are listed in Table \ref{table:databases}. All
databases supported searches using boolean search commands constructed from the
facets and keywords. The search yielded NNNN unique matches.

\begin{table}
    \begin{tabular}{ p{0.22\textwidth} p{0.76\textwidth} }
        \toprule
        Facet                  & Keywords   \\
        \midrule
        Agile methods\newline (before \& after) &
            agile, scrum, "extreme programming",\newline
            waterfall, "plan-driven", RUP \\
        Organizational transformation &
            transform*, transiti*, migrat*, journey, adopt*, deploy, introduc*,
            "roll-out", rollout \\
        Limit to software\newline related atricles &
            (software OR (conference="agile, xp, icgse, icse"))\newline
            NOT title+abs="manufacturing", conference="agile manufacturing"
        \\
        \bottomrule
    \end{tabular}
    \caption{Facets and related search terms used}
    \label{table:searchterms}
\end{table}

\begin{table}
    \begin{tabular}{ p{0.28\textwidth} l l }
        \toprule
        Database      & URL                     & N of matches   \\
        \midrule
        IEEExplore    & http://ieeexplore.ieee.org      & 745 \\ 
        ACM           & http://dl.acm.org               & 168 \\
        Scopus        & http://www.scopus.com/home.url  & 1596 \\
        Web of Knowledge\newline(Thomson Reuters) &
        http://apps.webofknowledge.com & 786 \\
        \bottomrule
    \end{tabular}
    \caption{Databases included in search}
    \label{table:databases}
\end{table}


The reason for deviating stydy assessment
--> Original research consisted almost solely of experience reports
--> The fact that experience reports have a subjective standpoint makes it
    unreliable to make quantitative conclusions based on the data
--> We also decided to include separate studies describing the same organization.
    This was useful as it deepened the understanding of each case, and several
    large organizations have produced more than one description of their agile
    transformation.

The fact that the study is based on a quantitative basis
--> Experie reports have a subjective bias
--> The data synthesis must not rely on quantitative measures. This means that
    we also considered topics raised by only a marginal number of reports.
--> It may still be interesting to make some quantitative observations such as
    a particular problem being reported in the majority of reports.


Data extraction was implemented partially using data extraction forms, but also
by a iterative coding approach. (The reason was? ...). Questions for which the
answer was easy to trace (for example, organization size, time of
transformation, and RQ1 why are transformations initiated) were answered by
filling in a data extraction form. For the more complex research questions (RQ2
how does transformation usually proceed, RQ3 which are typical success factors
and challenges) the iterative coding approach was used. The iterative coding
approach...
--> Possibility to find many different viewpoinits for the research questions
--> Lots of items, which would have lead to much diversity in data extraction form 


The primary research for this literature review consists almost exclusively of
industry experience reports. This may be seen as a factor lessening the
evidence.
--> Why did we do this anyway?
--> Case studies with research method studying organizational transformations in
    software industry are very scarce.
--> Large organizational changes are hard to measure experimentally (controlled
    experiments?). Also due to this any student research was excluded.

% ---------------------------------------------------------------------
\section{Results}
\label{sec:results}


% ---------------------------------------------------------------------
\section{Discussion}
\label{sec:discussion}


% ---------------------------------------------------------------------
\section{Limitations and future research}
\label{sec:conclusion}


% ---------------------------------------------------------------------
% -------------- BIBLIOGRAPHY -----------------------------------------
% ---------------------------------------------------------------------

\bibliographystyle{elsarticle-harv}
%\bibliographystylegen{plain}

\bibliography{sources_background}

\end{document}
