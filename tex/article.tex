% ---------------------------------------------------------------------
% -------------- PREAMBLE ---------------------------------------------
% ---------------------------------------------------------------------

%\documentclass[12pt,a4paper,finnish,oneside]{article}
\documentclass[lnbip]{svmultln}

\usepackage[utf8]{inputenc}
\usepackage[english]{babel}

\usepackage{calc}      % käytetään laskurien (counter) yhteydessä (tiedot.tex)
\usepackage{natbib}
\usepackage{url}
\usepackage{listings}
\usepackage{hyphenat}
\usepackage{supertabular,array}
\usepackage{lipsum}

% Strikethrough
\usepackage[normalem]{ulem}

% Punctuation for references
\bibpunct{[}{]}{;}{n}{,}{,}

% correct bad hyphenation here
\hyphenation{op-tical net-works semi-conduc-tor}


% ---------------------------------------------------------------------
% -------------- DOCUMENT ---------------------------------------------
% ---------------------------------------------------------------------

\begin{document}

\selectlanguage{english}

\mainmatter

\title{Large scale agile transformations: A systematic literature review}

\titlerunning{Large scale agile transformations}

\author{Kim Dikert \and Maria Paasivaara \and Casper Lassenius}
\authorrunning{Kim Dikert et al.}   % abbreviated author list (for running head)

% list of authors for the TOC (use if author list has to be modified)
\tocauthor{Kim Dikert, Maria Paasivaara, Casper Lassenius}

\institute{Aalto University . . . Address . . .
\\ \email{email@xx},
\\ WWW home page: \texttt{http://www...} }

\maketitle


% ---------------------------------------------------------------------
% -------------- ABSTRACT AND TEXT ------------------------------------
% ---------------------------------------------------------------------

\begin{abstract}        % give a summary of your paper
The abstract should summarize the contents of the paper
using at least 70 and at most 150 words. It will be set in 9-point
font size and be inset 1.0 cm from the right and left margins.
There will be two blank lines before and after the Abstract.
%                         please supply keywords within your abstract
\keywords {agile, transformation, large scale}
\end{abstract}


% ---------------------------------------------------------------------
\section{Introduction}

As the competition in software industry is growing companies are constantly
looking to improve their effectiveness. Agile methods are claimed to increase
productivity and quality \cite{Livermore2008}, which makes them attractive for
companies pursuing better performance. However, large organizations may have
problems introducing agile methods \cite{Dyba2009}. As agile methods are
initially designed for small teams their application has proven to be
problematic in a larger scale \cite{Boehm2005}.

It is typical for large companies to function according to traditional software
engineering models [Cite?!]. These models strive for consistent quality and performance
by rigorous planning and definition of process. This kind of approach is however
badly suited for software development, as projects typically encounter
situations that are too hard or impossible to foretell \cite{Schwaber2002}.
The main problems in plan-driven development are the high cost of changes and
late feedback on quality \cite{Petersen2010}. Long release cycles, cost of
resopnding to change and distance from customers undermine the  competitiveness
of companies. Agile methods are believed to bring remedy to these problems.

The motivation for this research is to 

-- Motivation: a gap in research

\lipsum[1]


% ---------------------------------------------------------------------
\section{Background}

\lipsum[1]

\subsection{Agile software development}

\lipsum[1]

\subsection{Introducing agile methods to an organization}

\lipsum[1]


% ---------------------------------------------------------------------
\subsection{Research method}

Research questions:

What are the motivations for starting an organizational transformation?

What kind of agile transformations have been reported?

What were challenges and success factors in the transformation process?

\subsection{Inclusion criteria}

\lipsum[1]

\subsection{Data sources and search strategy}

\lipsum[1]

\subsection{Final selection}

-- Figure of selection process

\lipsum[1]

\subsection{Data extraction and synthesis}

\lipsum[1]


% ---------------------------------------------------------------------
\section{Results}

More plain text.

\subsection{Overview of studies}

\lipsum[1]

\subsection{Overview of agile methods used}

\lipsum[1]

\subsection{Motivations for transformation}

\lipsum[1]


% ---------------------------------------------------------------------
\section{Conclusions}

-- Research on large scale agile transformations has been published.

-- Papers presenting the effect of agile on specific areas in software
development exist, but the whole has not been studied as much.

\subsection{Limitations}

-- The papers are mostly experience reports with no research method.

-- Some papers were of low quality, but they were included in order to give a
view of the topic that is broader and inclined towards practice.

-- Effect of organizational change is hard (or maybe impossible?) to measure
(kuulemma Marjo ja Kristian tietävät asiasta).

\subsection{Future research}

-- Very little studies on quantitative comparison of before/after transformation
has been done. (how about quantitative?)

-- What thihgs are important to report in an agile transformation case study?

-- Do not focus exclusively on the software perspective, but find parallels from
general research on organizaional transformation and leadership. (Tähän pitäs
vissiin sitte olla joku konkreettinen ehdotus?)

\lipsum[1]


% ---------------------------------------------------------------------
% -------------- BIBLIOGRAPHY -----------------------------------------
% ---------------------------------------------------------------------

\bibliographystyle{plain}

\bibliography{sources}


\end{document}
