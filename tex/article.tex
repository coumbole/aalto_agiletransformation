% ---------------------------------------------------------------------
% -------------- PREAMBLE ---------------------------------------------
% ---------------------------------------------------------------------

%\documentclass[12pt,a4paper,finnish,oneside]{article}
\documentclass[lnbip]{svmultln}

\usepackage[utf8]{inputenc}
\usepackage[english]{babel}

\usepackage{calc}      % käytetään laskurien (counter) yhteydessä (tiedot.tex)
\usepackage{natbib}
\usepackage{url}
\usepackage{listings}
\usepackage{hyphenat}
\usepackage{supertabular,array}
\usepackage{lipsum}
\usepackage{booktabs}

% Strikethrough
\usepackage[normalem]{ulem}

% Punctuation for references
\bibpunct{[}{]}{;}{n}{,}{,}

% correct bad hyphenation here
\hyphenation{op-tical net-works semi-conduc-tor}


% ---------------------------------------------------------------------
% -------------- DOCUMENT ---------------------------------------------
% ---------------------------------------------------------------------

\begin{document}

\selectlanguage{english}

\mainmatter

\title{Large scale agile transformations: A literature review}

\titlerunning{Large scale agile transformations}

\author{Kim Dikert \and Maria Paasivaara \and Casper Lassenius}
\authorrunning{Kim Dikert et al.}   % abbreviated author list (for running head)

% list of authors for the TOC (use if author list has to be modified)
\tocauthor{Kim Dikert, Maria Paasivaara, Casper Lassenius}

\institute{Aalto University . . . Address . . .
\\ \email{email@xx},
\\ WWW home page: \texttt{http://www...} }

\maketitle


% ---------------------------------------------------------------------
% -------------- ABSTRACT AND TEXT ------------------------------------
% ---------------------------------------------------------------------

\begin{abstract}        % give a summary of your paper
The abstract should summarize the contents of the paper
using at least 70 and at most 150 words. It will be set in 9-point
font size and be inset 1.0 cm from the right and left margins.
There will be two blank lines before and after the Abstract.
%                         please supply keywords within your abstract
\keywords {agile, transformation, large scale}
\end{abstract}


% ---------------------------------------------------------------------
\section{Introduction}

As the competition in software industry is growing companies are constantly
looking to improve their effectiveness. Agile methods are claimed to increase
productivity and quality \cite{Livermore2008}, which makes them attractive for
companies pursuing better performance. However, large organizations may have
problems introducing agile methods \cite{Dyba2009}. As agile methods are
initially designed for small teams their application has proven to be
problematic in a larger scale \cite{Boehm2005}.

It is typical for large companies to function according to traditional software
engineering models [Cite?!] (also why?). These models strive for consistent quality and performance
by rigorous planning and definition of process. This kind of approach is however
badly suited for software development, as projects typically encounter
situations that are too hard or impossible to foretell \cite{Schwaber2002}.
The main problems in plan-driven development are the high cost of changes and
late feedback on quality \cite{Petersen2010}. Long release cycles, cost of
resopnding to change and distance from customers undermine the  competitiveness
of companies. Agile methods are believed to bring remedy to these problems.

The goal of this research is to explore how and why large organizations adopt
agile methods. Studies on large scale agile transformations have been published,
but no extensive and summarizing research is available on the subject. In this
research we will give an overview of reported case studies on adopting agile
methods in large organizations. We will further discuss the motivations for
adopting agile methods, the characteristics of agile transformations, and
factors relating to success or challenge in the transformation process. Our
research is focused on the process of organizational change. We leave out topics
related to comparing the initial and transformed states of organisations, such
as measuring the performance of organizations before and after the
transformation.

This paper is organized as follows. The next section gives an overview of large
scale agile development. Section \ref{sec:method} presents the literature review
method used for conducting the research. The findings of the review are
discussed in section \ref{sec:results}, and section \ref{sec:conclusions}
discusses the findings and gives proposals for future research.


% ---------------------------------------------------------------------
\section{Background}
\label{sec:background}


-- What is agile

-- Differences to other methods?? (see d \& d 2008??)

-- Popular agile methods: Scrum, XP, and also Lean 


\subsection{Agile software development}

\lipsum[1]

\subsection{Introducing agile methods to an organization}

\lipsum[1]


% ---------------------------------------------------------------------
\section{Research method}
\label{sec:method}

The literature review was conducted as an application of Kitchenham's
method for systematic literature review (SLR) \cite{Kitchenham2007}. The method
was applied to be performed by only one researcher, and to serve as a
preliminary work evaluating the feasibility of a more in-depth review. The
required effort was limited by selecting only the following subset of SLR steps:
identification of primary sources, data extraction and synthesis. Steps left out
were the development of a systematic review protocol, strict study quality
assessment, and reviews in the study selection process.

The searches for primary sources was conducted in two steps. First a preliminary
search was made to sample the existing literature. Then a systematic search was
done based on the preliminary search. The following databases were included in
the searches: IEEExplore\footnote{http://ieeexplore.ieee.org/},
ACM\footnote{http://dl.acm.org/}, Scopus\footnote{http://www.scopus.com/},
ProQuest\footnote{http://search.proquest.com/}.
Additionally the archive of International XP
Conference\footnote{http://link.springer.com/} was searched.

The search phrases used in the preliminary search were \textit{agile
transformation} and \textit{large scale agile}. For the systematic search three
facets and appropriate keywords were chosen, as presented in Table
\ref{table:searchterms}. To search the databases a query phrase was constructed
by joining the keywords of each facet by \texttt{OR} operators and the facets by
\texttt{AND} operators.

\begin{table}[h]
    \centering
    \begin{tabular}{ll}
        \toprule
        Facet                  & Keywords   \\ \midrule
        Agile methods          & agile, scrum, lean, xp \\ 
        Organizational change  & transformation, transition, change, migration \\
        Large organization     & enterprise, organization, (large \texttt{AND} scale) \\
        \bottomrule
    \end{tabular}
    \caption{Facets and related search terms used}
    \label{table:searchterms}
\end{table}


% Research questions:
%
% What are the motivations for starting an organizational transformation?
%
% What kind of agile transformations have been reported?
%
% What were challenges and success factors in the transformation process?

\subsection{Inclusion criteria}

\lipsum[1]

\subsection{Data sources and search strategy}

\lipsum[1]

\subsection{Final selection}

-- Figure of selection process

\lipsum[1]

\subsection{Data extraction and synthesis}

\lipsum[1]


% ---------------------------------------------------------------------
\section{Results}
\label{sec:results}

More plain text.

\subsection{Overview of studies}

\lipsum[1]

\subsection{Overview of agile methods used}

\lipsum[1]

\subsection{Motivations for transformation}

\lipsum[1]


% ---------------------------------------------------------------------
\section{Conclusions}
\label{sec:conclusions}

-- Research on large scale agile transformations has been published.

-- Papers presenting the effect of agile on specific areas in software
development exist, but the whole has not been studied as much.

\subsection{Limitations}

-- The papers are mostly experience reports with no research method.

-- Some papers were of low quality, but they were included in order to give a
view of the topic that is broader and inclined towards practice.

-- Effect of organizational change is hard (or maybe impossible?) to measure
(kuulemma Marjo ja Kristian tietävät asiasta).

\subsection{Future research}

-- Very little studies on quantitative comparison of before/after transformation
has been done. (how about quantitative?)

-- What thihgs are important to report in an agile transformation case study?

-- Do not focus exclusively on the software perspective, but find parallels from
general research on organizaional transformation and leadership. (Tähän pitäs
vissiin sitte olla joku konkreettinen ehdotus?)

\lipsum[1]


% ---------------------------------------------------------------------
% -------------- BIBLIOGRAPHY -----------------------------------------
% ---------------------------------------------------------------------

\bibliographystyle{plain}

\bibliography{sources}


\end{document}
